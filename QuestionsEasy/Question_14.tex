\section{kD-деревья. Окто- и квадро-деревья.}

\subsection{kD-деревья}

ДОДЕЛАТЬ
\begin{definition}
	\textbf{kD-дерево} (k-dimensional tree) --- это структура данных, используемая для организации точек в k-мерном пространстве. Это древовидная структура, где каждый узел представляет собой k-мерную точку, а дерево рекурсивно разделяет пространство на две части гиперплоскостями, перпендикулярными осям координат.
\end{definition}

\noindent \textbf{Основные характеристики и применение:}
\begin{itemize}
	\item kD-деревья являются двоичными деревьями поиска.
	\item Они используются для эффективного поиска ближайших соседей и поиска в диапазоне в многомерных пространствах.
	\item kD-деревья находят применение в машинном обучении, компьютерной графике, базах данных и других областях, где требуется работа с многомерными данными.
\end{itemize}

\subsection{Октодеревья}

\begin{definition}
	\textbf{Октодерево} (octree) --- это древовидная структура данных, используемая для разделения трехмерного пространства. В октодереве каждый внутренний узел имеет восемь потомков, каждый из которых представляет собой октант (восьмую часть) пространства, представленного родительским узлом.
\end{definition}

\noindent \textbf{Основные характеристики и применение:}
\begin{itemize}
	\item Октодеревья являются трехмерным аналогом квадродеревьев.
	\item Они используются для пространственной индексации, обнаружения столкновений в 3D-графике, трассировки лучей и хранения воксельных данных.
	\item Октодеревья позволяют эффективно выполнять поиск объектов в трехмерном пространстве.
\end{itemize}

\subsection{Квадродеревья}

\begin{definition}
	\textbf{Квадродерево} (quadtree) --- это древовидная структура данных, в которой каждый внутренний узел имеет ровно четыре потомка. Квадродеревья используются для рекурсивного разбиения двумерного пространства на четыре квадранта.
\end{definition}

\noindent \textbf{Основные характеристики и применение:}
\begin{itemize}
	\item Квадродеревья являются двумерным аналогом октодеревьев.
	\item Они применяются для пространственной индексации, сжатия изображений, обнаружения столкновений в 2D-графике и в географических информационных системах (ГИС).
	\item Квадродеревья позволяют быстро находить объекты в двумерном пространстве и эффективно выполнять операции, связанные с пространственными данными.
\end{itemize}