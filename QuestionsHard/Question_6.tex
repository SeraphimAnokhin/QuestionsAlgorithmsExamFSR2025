\section{Потоки в сетях. Задача о максимальном потоке. Алгоритм Форда – Фалкерсона.}
Введём необходимые определения

\textbf{Сеть} $G = (V, E)$  --- ориентированный граф, в котором:
\begin{enumerate}
	\item каждое ребро $(u,v) \in E$
	имеет положительную пропускную способность $c(u, v) > 0$ (англ. capacity). Если $(u, v) \notin E$, то $c(u, v) = 0$,
	\item выделены две вершины --- \textbf{исток} $s$ и \textbf{сток} $s$.
\end{enumerate}

\textbf{Поток} (англ. flow) $f$ в $G$ --- действительная функция $f: V \times V \to \mathbb{R}$, удовлетворяющая условиям:

\begin{enumerate}
	\item $f(u, v) = -f(v, u)$ (антисимметричность);
	\item $f(u, v) \le c(u, v)$ (ограничение пропускной способности), если ребра нет, то $f(u, v) = 0$;
	\item $\sum_v f(u, v) = 0$ для всех вершин $u$, кроме $s$ и $t$ (закон сохранения потока).
\end{enumerate}

\noindent \textbf{Величина потока} $f$ определяется как $\displaystyle |f| = \sum_{v \in V} f(s, v) = \sum_{v \in V} f(v,t)$.

В задаче о \textbf{нахождении максимального потока} дана некоторая сеть $G$ с истоком $s$ и стоком $t$ и требуется найти поток максимальной величины.

 
\subsection*{Алгоритм Форда-Фалкерсона}

\begin{verbatim}
ddd
dddd
\end{verbatim}




