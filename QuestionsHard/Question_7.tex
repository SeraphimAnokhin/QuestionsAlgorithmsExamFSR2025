\section{Амортизационный анализ: групповой анализ, банковский метод. Амортизационный анализ для бинарного счётчика }

\textbf{Амортизационный анализ:}
Метод анализа производительности алгоритмов, учитывающий общую стоимость последовательности операций, а не только стоимость одной.
Идея:
\begin{itemize}
	\item Пересчитать на более скрытые или редкие операции.
	\item Стоимость операции как бы <<размазывается>> на все операции, входящие в последовательность.
	\item Отражает наихудший случай.
	\item Не учитывает вероятность.
\end{itemize}

\textbf{Метод группировки:}
В случае, когда в наихудшем случае суммарное время выполнения $k$ операций $T_k(N)$ не зависит от $N$, то амортизированная стоимость каждой операции $A(N) = T_k(N)/k$.

\subsection*{Пример: Бинарный счетчик}
Счетчик имеет $n$ бит (например, $N$ значений).
Поддерживает операции $Increment$ и $Reset$.
Пусть $N=5$. Изначально счетчик равен 00000.
Пример:\\
0. 00000\\
1. 00001\\
2. 00010\\
3. 00011\\
4. 00100\\

За $N$ элементов каждый раз $k$ операций. $k$ - количество операций.
Например, для $N=4$:
000 $\rightarrow$ 001 (1 раз)
001 $\rightarrow$ 010 (2 раза)
010 $\rightarrow$ 011 (1 раз)
011 $\rightarrow$ 100 (3 раза)
$\Rightarrow$ $N$ бит менялись $k$ раз.

Тогда амортизированная стоимость $O(k)$, где $k=2N$.
$A(N) = N + \frac{N}{2} + \frac{N}{4} + \dots + 1 = 2N$.
Сложность $O(1)$.

\subsection*{Бухгалтерский метод (взвешивания)}
1. На операции тратится столько условных единиц, сколько больше (меньше) реально.

2. Если уж стоит и реально получится, то разность сил и цен на $k$-ю операцию.

3. Условные деньги вводится так, чтобы их хватало на любые операции. Это для того, чтобы кредит не были отрицательными деньги.
