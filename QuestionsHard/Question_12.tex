\section{Сканирующая прямая. Алгоритм Бентли-Оттоманна для поиска пересечения отрезков}

\textbf{Сканирующая прямая} --- прямая, проходящая по точкам (отсортированным по какому-то признаку, например, по координате). Пример: в автомате $S_v$ по координатам. 

\begin{center}
	\textbf{Характеристические положения}
\end{center}
\begin{tabular}{|l|l|}
	\hline
	\textbf{Положение} & \textbf{Действие / Результат} \\
	\hline
	Начало отрезка X & $\implies$ +2 новых пар соседних отрезков  \\
	\hline
	Конец отрезка & $\implies$ +1 новая пара отрезков  \\
	\hline
	Точка пересечения отрезков & $\implies$ +2 пары  \\
	\hline
\end{tabular}

\begin{figure}[h!]
	\centering
	% Placeholder for the image. You'd replace this with your actual image file.
	% \includegraphics[width=0.8\linewidth]{path/to/your/image.png}
	\caption*{Визуализация характерных положений (из исходного документа)} % Optional, if you want a descriptive caption without "Figure X"
\end{figure}

\subsection{Наивный алгоритм}

\textbf{Задача:} поиск точек пересечения пар отрезков на плоскости. 

\textbf{Входные данные:}
\begin{itemize}
	\item пары начал отрезков
	\item пары концов отрезков
\end{itemize}

\textbf{Выходные данные:} точки пересечения с указанием пар отрезков. 

\textbf{Алгоритм:} Проверки на пересечение каждого отрезка. 

\textbf{Сложность:} $O(n^2)$. 

\subsection{Алгоритм Бентли --- Оттмана}

Пусть:
\begin{itemize}
	\item нет вертикальных отрезков 
	\item пересекается не более чем 11 отрезков (возможно, опечатка, 11 - необычное число, обычно имеется в виду константа) 
	\item Начало (конец) отрезка не является точкой пересечения. 
\end{itemize}
(данные ограничения необязательны, просто при их наличии намного проще формулировать алгоритм --- при необходимости каждое ограничение можно убрать)

\textbf{Усовершенствуем наивный алгоритм.}
\begin{enumerate}
	\item Если отрезки не были соединены (в смысле порядка) по $X$ точек пересечения отрезков, то они не могут пересечься. 
	\item Характерные положения заметяющей прямой определяют соседние отрезки (события). 
\end{enumerate}

\textbf{Определим события:} 
\begin{itemize}
	\item Начало отрезка $\implies$ добавление пары 
	\item Конец отрезка $\implies$ удаление пары 
	\item Пересечение т.ч. $\implies$ поменять местами 
\end{itemize}
В течение некоторого периода остаются постоянными, что приводит к тому, что прямая может двигаться дискретно. 

События будем хранить в очереди с приоритетом (PQ), которая движется слева направо (min по координате). 

Отрезки будем хранить в бинарном дереве поиска, т.к. необходимо вычислять соседей (BT). 

\subsection*{Алгоритм.}

\textbf{Инициализация:} $BT = \emptyset$, $Ans = \emptyset$ (структура для ответа). 
\begin{itemize}
	\item PQ заполнена точками начал и концов отрезков. 
\end{itemize}

\textbf{Шаг:} Пока PQ не пусто, извлекаем событие $e$. 

\begin{enumerate}[label=\arabic*.]
	\item \textbf{$e$ --- начало отрезка $a$} 
	\begin{itemize}
		\item Добавляем $a$ в BT. 
		\item Находим соседей $a$ в дереве BT: $t, b$. Проверяем на пересечение $a, t$ и $a, b$. 
		\item Добавляем событие в PQ, и если есть пересечение, то добавляем пары в $Ans$. 
	\end{itemize}
	\item \textbf{$e$ --- конец отрезка $a$} 
	\begin{itemize}
		\item Находим соседей отрезка $a$ в дереве BT: $b, t$. 
		\item Удаляем $a$ из BT. 
		\item Проверяем пересечение $b$ и $t$. 
		\item Добавляем событие в PQ, и если есть пересечение, то добавляем пары в $Ans$. 
	\end{itemize}
	\item \textbf{$e$ --- пересечение отрезков $a, b$} 
	\begin{itemize}
		\item Находим верхнего соседа $a$ ($t$) и нижнего соседа $b$ ($k$) в BT. 
		\item Меняем $a$ и $b$ в BT местами. 
		\item Проверяем пересечение пары $(t, b)$ и пары $(a, k)$. 
		\item Добавляем событие в PQ, и если есть пересечение, то добавляем пары в $Ans$. 
	\end{itemize}
\end{enumerate}
При вставке можно удалять событие $(t, b)$. 
При смене порядка можно удалять событие $(a, t)$. 

\textbf{Сложность алгоритма:} $O(n \log n)$ 
\begin{itemize}
	\item Сортировка событий: $O(n \log n)$ 
	\item Поиск/удаление/вставка в дерево: $O(\log n)$ (в случае событий) 
	\item Проверка на пересечение: $O(\log n)$ (в случае событий) 
	\item Всего событий $n$ $\Rightarrow$ $O(n \log n)$
\end{itemize}
Итого: $O(n \log n)$ 
