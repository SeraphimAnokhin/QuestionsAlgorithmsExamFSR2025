\section{Алгоритм Бойера-Мура. Эвристики стоп-символа и хорошего суффикса.}

\textbf{Цель:} найти все подстроки \textsf{str2} в \textsf{str1}.

\textbf{Суть:} базовый алгоритм для поиска:
\begin{itemize}
	\item Проходимся по всем \textsf{str1} от $i$-го до $i+\text{len(str2)}-1$.
	\item Посимвольно сравниваем.
\end{itemize}
Модернизированный алгоритм: обходит некоторое количество символов.

\subsection*{Эвристика стоп-символа:}
\begin{itemize}
	\item Мы спускаемся в рамках $i$ окна слева-направо.
	\item И находим несовпадение в \textsf{str2}.
	\item Из того, что мы знаем про \textsf{str2} (какой символ там должен быть), смотрим на $str1[i+k]$ и $str2[k]$.
	\item И если $str1[i+k]$ совпадает с символом в $str2$, то сдвигаем окно так, чтобы $str1[i+k]$ совпал с этим символом. Иначе сдвигаем окно на длину \textsf{str2}.
\end{itemize}

Пример:
\textsf{str1}: \texttt{abcabcab}
\textsf{str2}: \texttt{abab}

В этом случае сравнения $str1[3]$ и $str2[3]$ дадут "c" и "b" соответственно.
\textsf{str1} \texttt{...c...}
\textsf{str2} \texttt{...b...}

Как это влияет на наши дальнейшие действия?
Мы в $str1[3]$ видим "с". Если в $str2$ нет "с", то мы сдвигаем окно на $length(str2)$, начиная с $i+k$.

\subsection*{Эвристика хорошего суффикса:}
Мы находим подстроку в \textsf{str2}, которая совпадает с подстрокой в \textsf{str1}, то есть $str1[i+k \dots i+\text{len(str2)}-1]$ совпадает с $str2[k \dots \text{len(str2)}-1]$.
Это называется "хороший суффикс".
Дальше мы ищем повторение "хорошего суффикса" в $str2$.

Пример: \textsf{str1} \texttt{абабаб} ищем \texttt{абаб}.
\begin{verbatim}
	str1: абабаб
	str2: абаб
\end{verbatim}
Шаги:
\begin{verbatim}
	str1: абабаб
	str2:  абаб
\end{verbatim}

\begin{enumerate}
	\item Во-вторых, мы можем попытаться присоединить \textsf{str2} к \textsf{str1} так, чтобы у нас совпали несколько символов.
	\item Из этого вытекает: $str1[i \dots i+\text{len(str2)}-1]$ должен совпадать $str2[0 \dots \text{len(str2)}-1]$.
\end{enumerate}
Пример:
\textsf{str1}: \texttt{aabbabcaba}
\textsf{str2}: \texttt{abab}